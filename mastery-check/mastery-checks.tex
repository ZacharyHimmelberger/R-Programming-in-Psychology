\documentclass[letterpaper,12pt]{article}

\usepackage{setspace}
\usepackage{listings}
\usepackage{amsmath}
\usepackage{setspace}
\usepackage{csquotes}
\usepackage[margin=1in]{geometry}

\doublespacing

\begin{document}

\section{Class 1 - June 1}

\subsection{Basics of R Syntax}

\begin{enumerate}
    \item Write a two line R script. 
    Make the first line a comment and the second line a simple addition problem. 
    Run the script to ensure that you get the expected result.
    \item Add a third line to the script from above. 
    This line should be another simple addition problem. 
    Run each of the three lines individually, then again as an entire script. 
    \item Explain why \verb|[1]| appears after running \verb|4+4|.
    \item Create an object called height and assign your height in inches as the value. 
    \item Convert the height object to centimeters ($1 \: in. \approx 2.54 \: cm$). 
    Assign the result to the object \verb|height|.
    \item Run \verb|height|. 
    Why did this print a value to the screen, when running the above command did not?
\end{enumerate}

\subsection{R Markdown}

\begin{enumerate}
    \item Create a pdf in R Markdown. 
    It should contain a brief explanation of your proposed final project idea. 
    It should also contain two R chunks. 
    The first chunk should assign a variable called \verb|normDist|, which contains 1000 numbers drawn from a normal distribution. 
    The second chunk should contain a graph of those numbers. 
\end{enumerate}

\end{document}